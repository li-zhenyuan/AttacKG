\section{Related Work}
\label{sec:relatedworks}

\subsection{Extract threat intelligence from unstructured CTI reports}

Existing works try to extract IoC \cite{Liao}, TTPs \cite{Husari2017}, Attack Chains \cite{Zhu2018}, Attack Graph \cite{Gao} from unstructed TI. Most recent works tend to extract more detailed and structure information. Ideal CTI should be general to cover more attack cases while detailed to avoid FP.  To strike a balance, we need to correlate related attack descriptions and construct a better one. 

\subsection{Threat detection and forensic (Adopt graphs to represent cyber attacks)}

\cite{Milajerdi2019}, etc. can adopt threat intelligence extracted by our system. Accurate and general attack description can ensure detection efficiency and accuracy. 

\subsection{Modeling the cyber threat intelligence (Knowledge graphs for security purpose)}

\cite{Gao2020,Zhao2020} try to model and quantify the underlying relationship among heterogeneous IoCs (Attackers, Device, Platform, Vulnerability, File, Type). These works do not include information about how these IoCs work together (PG), and leave lots of details. 

