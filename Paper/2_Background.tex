\section{Background}
\label{sec:background}

\subsection{Problem statement}

CTI reports contain rich information needed by detection and analyzer. However, existing reports analysis work focus only on the reports itself, leave out lots of useful information. We propose to adopt TTPs knowledge and Technique/Tactic Knowledge Graph(TKG) to organize the knowledge extracted from reports. And generate more useful attack descriptions.

On the one hand, current CTI sources focus on naïve IoCs, such as bad IPs, malware hashes, etc., without much high-level semantic. Such CTI are neither general nor reliable. \cite{Li2019} On the other hand, unstructed threat whitepapers are vague. – We need new CTI standards. 

On the other hand, effective(fast) and efficient(accuracy) detection require accurate attack description. State-of-the-art work rely on manual analysis \cite{Milajerdi2019} which is difficult to expand. – We need more and automated CTI. 

\subsection{Challenges}

C1. How to extract Techniques from Natural-Language CTI? – Contribution 1: Automated report parsing.

S: TTPs templates (knowledge base) vs. Patterns in CTI (NLP)

C1-1. Unstructured Threat Whitepapers Are Vague:

Vague nodes: Lack of explicit node identification; Vague subject: 

Vague edges: An operation may corresponds to a series of edges in PG

C1-2. How to find technique dependencies? (Sometimes)
S: Employ System Entity/Report/self-defined tags(TCP sockets) as connections

C2. How to integrate multiple CTI reports? – Contribution 2: Building attack Technique Knowledge Graph.

Observations:

1. A single report most likely covers a fragment of an APT attack?

2.Different reports may conflict due to variant malwares

S: Agree to disagree (Different Versions)

C2-1. How to design the TKG? – Including what kind of nodes (system entities, techniques), edges? 

C2-2. How to connect/match/cluster different reports and build the TKG?

C2-3. How to use the TKG?  - What basic function we need to implement on TKG?
